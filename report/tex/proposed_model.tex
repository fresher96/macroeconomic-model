
\section{Proposed Model}
This section describes the proposed model on a high-level.
The first subsection describes the economic indicators involved in the model.
The second subsection describes the ODEs system proposed to model the dynamics of these indicators.

Basically, the model is a set of differential equations that describe the inter-relationships between different macroeconomic indicators.
We are using the term \textit{macroeconomic indicators} to refer to the factors, features, or variables that we would like to model.
Such as annual GDP growth, unemployment rate, and similar indicators.
The proposed model is a system of ODEs that tries to capture the dynamics of such indicators.
Each equation in the system may involve:
an indicator value, a function of some indicator, a change (differential) of some indicator, and so on.
This will be clarified shortly in section \ref{sec:model}.

\subsection{Macroeconomic Indicators}
We classify the indicators used in our model into two categories:
(a) International indicator (items 1-5), and
(b) Russia-specific indicators (items 6-11).
\begin{enumerate}
\item Annual GDP
\item Base Interest Rate (from the central bank)
\item Inflation Rate (correlated with CPI)
\item Unemployment Rate
\item Average Wage (i.e average income per person)
\item USD/RUB (exchange rate of US dollar for ruble)
\item Brent Crude Oil Price
\item MOEX Equity Index
\item State-funded investments (per year)
\item Defense Spendings
\item Population Growth Rate
\end{enumerate}
We refer to these vectors using a vector
$\vec{x} = (x_1, \cdots, x_n)$
where in $n = 11$ and each $x_i$ refers to the corresponding item.
For example, $x_4$ is unemployment rate.

\subsection{Mathematical Model}
\label{sec:model}
The proposed model is encapsulated by the following equation which represents a system of stochastic differential equations (SDEs):
\begin{equation}
\label{eq:model}
d\vec{x} = A \vec{x} \, dt + \Sigma \, d\vec{W_t}
\end{equation}

$\vec{x}$  is defined in the previous section.
$A$ is an $n \times n$ matrix of constant coefficients (not dependent on $x$).
$t$ is time.
$\Sigma$ is an $n \times n$ matrix.
$\vec{W_t}$ is a vector of $n$ independent Brownian motion terms.

If we omit the second term for the moment,
by setting $\Sigma = 0$ for example.
We can see that the system is simply reduced to a linear system of ODEs.
The second term introduces stochasticity in our model by using Brownian motion terms.

$\Sigma$ corresponded to a correlation matrix.
We consider the value of $\Sigma^T \Sigma$ to be of the form $B(x)$.
That is, $\Sigma^T \Sigma = B(x)$,
Where $B(x)$ is an $n \times n$ matrix that is dependent on $x$.
We consider an element $B_{ij}(x)$ of this matrix to have the form
$B_{ij}(x) = b_{ij} \cdot x + c_{ij}$.
$b_{ij}$ is a vector of length $n$;
the dot $\cdot$ is the dot product;
and the components of the vector $b_{ij}$ along with $c_{ij}$ are constants independent of $x$.
This model is called an affine model \parencite{citation needed}.
By this formulation, it can be shown that $\Sigma = \sqrt{B}$ \parencite{citation needed}.

We start with a reduced simplified model where we set $b_{ij} = 0$.
This basically corresponds to setting $B$ as a constant matrix independent of $x$.
Which corresponds to $\Sigma$ being independent of $x$.
After the implementation and calibration of this reduced model,
we reconsider the original model if time is available.
